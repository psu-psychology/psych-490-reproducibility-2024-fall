% Options for packages loaded elsewhere
\PassOptionsToPackage{unicode}{hyperref}
\PassOptionsToPackage{hyphens}{url}
\PassOptionsToPackage{dvipsnames,svgnames,x11names}{xcolor}
%
\documentclass[
  letterpaper,
  DIV=11,
  numbers=noendperiod]{scrartcl}

\usepackage{amsmath,amssymb}
\usepackage{iftex}
\ifPDFTeX
  \usepackage[T1]{fontenc}
  \usepackage[utf8]{inputenc}
  \usepackage{textcomp} % provide euro and other symbols
\else % if luatex or xetex
  \usepackage{unicode-math}
  \defaultfontfeatures{Scale=MatchLowercase}
  \defaultfontfeatures[\rmfamily]{Ligatures=TeX,Scale=1}
\fi
\usepackage{lmodern}
\ifPDFTeX\else  
    % xetex/luatex font selection
\fi
% Use upquote if available, for straight quotes in verbatim environments
\IfFileExists{upquote.sty}{\usepackage{upquote}}{}
\IfFileExists{microtype.sty}{% use microtype if available
  \usepackage[]{microtype}
  \UseMicrotypeSet[protrusion]{basicmath} % disable protrusion for tt fonts
}{}
\makeatletter
\@ifundefined{KOMAClassName}{% if non-KOMA class
  \IfFileExists{parskip.sty}{%
    \usepackage{parskip}
  }{% else
    \setlength{\parindent}{0pt}
    \setlength{\parskip}{6pt plus 2pt minus 1pt}}
}{% if KOMA class
  \KOMAoptions{parskip=half}}
\makeatother
\usepackage{xcolor}
\ifLuaTeX
  \usepackage{luacolor}
  \usepackage[soul]{lua-ul}
\else
  \usepackage{soul}
  
\fi
\setlength{\emergencystretch}{3em} % prevent overfull lines
\setcounter{secnumdepth}{-\maxdimen} % remove section numbering
% Make \paragraph and \subparagraph free-standing
\makeatletter
\ifx\paragraph\undefined\else
  \let\oldparagraph\paragraph
  \renewcommand{\paragraph}{
    \@ifstar
      \xxxParagraphStar
      \xxxParagraphNoStar
  }
  \newcommand{\xxxParagraphStar}[1]{\oldparagraph*{#1}\mbox{}}
  \newcommand{\xxxParagraphNoStar}[1]{\oldparagraph{#1}\mbox{}}
\fi
\ifx\subparagraph\undefined\else
  \let\oldsubparagraph\subparagraph
  \renewcommand{\subparagraph}{
    \@ifstar
      \xxxSubParagraphStar
      \xxxSubParagraphNoStar
  }
  \newcommand{\xxxSubParagraphStar}[1]{\oldsubparagraph*{#1}\mbox{}}
  \newcommand{\xxxSubParagraphNoStar}[1]{\oldsubparagraph{#1}\mbox{}}
\fi
\makeatother


\providecommand{\tightlist}{%
  \setlength{\itemsep}{0pt}\setlength{\parskip}{0pt}}\usepackage{longtable,booktabs,array}
\usepackage{calc} % for calculating minipage widths
% Correct order of tables after \paragraph or \subparagraph
\usepackage{etoolbox}
\makeatletter
\patchcmd\longtable{\par}{\if@noskipsec\mbox{}\fi\par}{}{}
\makeatother
% Allow footnotes in longtable head/foot
\IfFileExists{footnotehyper.sty}{\usepackage{footnotehyper}}{\usepackage{footnote}}
\makesavenoteenv{longtable}
\usepackage{graphicx}
\makeatletter
\def\maxwidth{\ifdim\Gin@nat@width>\linewidth\linewidth\else\Gin@nat@width\fi}
\def\maxheight{\ifdim\Gin@nat@height>\textheight\textheight\else\Gin@nat@height\fi}
\makeatother
% Scale images if necessary, so that they will not overflow the page
% margins by default, and it is still possible to overwrite the defaults
% using explicit options in \includegraphics[width, height, ...]{}
\setkeys{Gin}{width=\maxwidth,height=\maxheight,keepaspectratio}
% Set default figure placement to htbp
\makeatletter
\def\fps@figure{htbp}
\makeatother
% definitions for citeproc citations
\NewDocumentCommand\citeproctext{}{}
\NewDocumentCommand\citeproc{mm}{%
  \begingroup\def\citeproctext{#2}\cite{#1}\endgroup}
\makeatletter
 % allow citations to break across lines
 \let\@cite@ofmt\@firstofone
 % avoid brackets around text for \cite:
 \def\@biblabel#1{}
 \def\@cite#1#2{{#1\if@tempswa , #2\fi}}
\makeatother
\newlength{\cslhangindent}
\setlength{\cslhangindent}{1.5em}
\newlength{\csllabelwidth}
\setlength{\csllabelwidth}{3em}
\newenvironment{CSLReferences}[2] % #1 hanging-indent, #2 entry-spacing
 {\begin{list}{}{%
  \setlength{\itemindent}{0pt}
  \setlength{\leftmargin}{0pt}
  \setlength{\parsep}{0pt}
  % turn on hanging indent if param 1 is 1
  \ifodd #1
   \setlength{\leftmargin}{\cslhangindent}
   \setlength{\itemindent}{-1\cslhangindent}
  \fi
  % set entry spacing
  \setlength{\itemsep}{#2\baselineskip}}}
 {\end{list}}
\usepackage{calc}
\newcommand{\CSLBlock}[1]{\hfill\break\parbox[t]{\linewidth}{\strut\ignorespaces#1\strut}}
\newcommand{\CSLLeftMargin}[1]{\parbox[t]{\csllabelwidth}{\strut#1\strut}}
\newcommand{\CSLRightInline}[1]{\parbox[t]{\linewidth - \csllabelwidth}{\strut#1\strut}}
\newcommand{\CSLIndent}[1]{\hspace{\cslhangindent}#1}

\KOMAoption{captions}{tableheading}
\makeatletter
\@ifpackageloaded{tcolorbox}{}{\usepackage[skins,breakable]{tcolorbox}}
\@ifpackageloaded{fontawesome5}{}{\usepackage{fontawesome5}}
\definecolor{quarto-callout-color}{HTML}{909090}
\definecolor{quarto-callout-note-color}{HTML}{0758E5}
\definecolor{quarto-callout-important-color}{HTML}{CC1914}
\definecolor{quarto-callout-warning-color}{HTML}{EB9113}
\definecolor{quarto-callout-tip-color}{HTML}{00A047}
\definecolor{quarto-callout-caution-color}{HTML}{FC5300}
\definecolor{quarto-callout-color-frame}{HTML}{acacac}
\definecolor{quarto-callout-note-color-frame}{HTML}{4582ec}
\definecolor{quarto-callout-important-color-frame}{HTML}{d9534f}
\definecolor{quarto-callout-warning-color-frame}{HTML}{f0ad4e}
\definecolor{quarto-callout-tip-color-frame}{HTML}{02b875}
\definecolor{quarto-callout-caution-color-frame}{HTML}{fd7e14}
\makeatother
\makeatletter
\@ifpackageloaded{caption}{}{\usepackage{caption}}
\AtBeginDocument{%
\ifdefined\contentsname
  \renewcommand*\contentsname{Table of contents}
\else
  \newcommand\contentsname{Table of contents}
\fi
\ifdefined\listfigurename
  \renewcommand*\listfigurename{List of Figures}
\else
  \newcommand\listfigurename{List of Figures}
\fi
\ifdefined\listtablename
  \renewcommand*\listtablename{List of Tables}
\else
  \newcommand\listtablename{List of Tables}
\fi
\ifdefined\figurename
  \renewcommand*\figurename{Figure}
\else
  \newcommand\figurename{Figure}
\fi
\ifdefined\tablename
  \renewcommand*\tablename{Table}
\else
  \newcommand\tablename{Table}
\fi
}
\@ifpackageloaded{float}{}{\usepackage{float}}
\floatstyle{ruled}
\@ifundefined{c@chapter}{\newfloat{codelisting}{h}{lop}}{\newfloat{codelisting}{h}{lop}[chapter]}
\floatname{codelisting}{Listing}
\newcommand*\listoflistings{\listof{codelisting}{List of Listings}}
\makeatother
\makeatletter
\makeatother
\makeatletter
\@ifpackageloaded{caption}{}{\usepackage{caption}}
\@ifpackageloaded{subcaption}{}{\usepackage{subcaption}}
\makeatother

\ifLuaTeX
  \usepackage{selnolig}  % disable illegal ligatures
\fi
\usepackage{bookmark}

\IfFileExists{xurl.sty}{\usepackage{xurl}}{} % add URL line breaks if available
\urlstyle{same} % disable monospaced font for URLs
\hypersetup{
  pdftitle={The reproducibility crisis in science},
  colorlinks=true,
  linkcolor={blue},
  filecolor={Maroon},
  citecolor={Blue},
  urlcolor={Blue},
  pdfcreator={LaTeX via pandoc}}


\title{The reproducibility crisis in science}
\usepackage{etoolbox}
\makeatletter
\providecommand{\subtitle}[1]{% add subtitle to \maketitle
  \apptocmd{\@title}{\par {\large #1 \par}}{}{}
}
\makeatother
\subtitle{PSYCH 490.012}
\author{}
\date{2024-09-06}

\begin{document}
\maketitle


\subsection{Fall 2024 • MWF 3:35-4:20
PM}\label{fall-2024-mwf-335-420-pm}

\subsection*{The reproducibility crisis in
science}\label{the-reproducibility-crisis-in-science}
\addcontentsline{toc}{subsection}{The reproducibility crisis in science}

Much attention has focused on the reproducibility of research in
psychology, but the challenges of producing robust and reliable
knowledge extend to all disciplines, not just in science. In this
seminar, we will discuss whether there is or is not a reproducibility
crisis in psychology and in science more broadly. We will discuss how
initiatives to make scientific research more open and transparent can
also make it more reproducible and robust.

\subsection*{Instructor}\label{instructor}
\addcontentsline{toc}{subsection}{Instructor}

Rick O. Gilmore, Ph.D.~\\
Professor of Psychology\\
rog1 AT-SIGN psu PERIOD edu\\

Schedule an appointment:
\url{https://doodle.com/mm/rickgilmore/book-a-time}

Lab web site: \url{https://gilmore-lab.github.io}

\subsection*{Teaching Assistant}\label{teaching-assistant}
\addcontentsline{toc}{subsection}{Teaching Assistant}

Adam Calderon\\
Graduate Student in Clinical Psychology\\
afc6160 AT-SIGN psu PERIOD edu

\subsection*{Meeting time \& location}\label{meeting-time-location}
\addcontentsline{toc}{subsection}{Meeting time \& location}

Monday, Wednesday, \& Friday, 3:35 PM - 4:20 PM\\
\href{https://map.psu.edu/?id=1134\#!ct/59509,33177,25403,26748,26749,26750,27255?m/357035?s/cedar}{Cedar
Building} 134

\subsection{Canvas site}\label{canvas-site}

We will use Canvas to submit assignments and grade them. The Canvas site
may be found here: \url{https://psu.instructure.com/courses/2350148}.

Most of the course content will be found on this site.

\subsection{Course structure}\label{course-structure}

This is a discussion-focused course. On most days we will discuss
readings assigned prior to class. On many Fridays and a few Mondays, we
will work together or individually on the assigned
\href{index.qmd\#exercises}{exercises}, the final project, or another
assignment.

\subsection{Schedule}\label{schedule}

\subsubsection*{August 26-30}\label{week-01}
\addcontentsline{toc}{subsubsection}{August 26-30}

\paragraph*{Monday, August 26}\label{monday-august-26}
\addcontentsline{toc}{paragraph}{Monday, August 26}

\emph{Introduction to the course: Why trust science?}

\begin{itemize}
\tightlist
\item
  Read

  \begin{itemize}
  \tightlist
  \item
    (Recommended) (Oreskes 2019, chap. 1, pp.~55-68) \textbar{}
    \href{https://psu.instructure.com/courses/2350148/files/folder/readings?preview=165170717}{PDF
    on Canvas}.
  \end{itemize}
\item
  (Extra credit) Complete
  \href{https://forms.gle/471gpq6XXXq9iG9C8}{Survey 01 on Trust in
  Science and Scientists}
\item
  \href{notes/wk01-2024-08-26-intro.qmd}{Class notes}
\end{itemize}

\begin{tcolorbox}[enhanced jigsaw, left=2mm, leftrule=.75mm, title=\textcolor{quarto-callout-tip-color}{\faLightbulb}\hspace{0.5em}{Tip}, coltitle=black, rightrule=.15mm, arc=.35mm, toprule=.15mm, colbacktitle=quarto-callout-tip-color!10!white, toptitle=1mm, colframe=quarto-callout-tip-color-frame, breakable, bottomtitle=1mm, opacityback=0, bottomrule=.15mm, titlerule=0mm, opacitybacktitle=0.6, colback=white]

To earn 3 extra credit points for completing the survey, send the TA one
of the following:

\begin{enumerate}
\def\labelenumi{\arabic{enumi}.}
\tightlist
\item
  The date and time you completed the survey in the following format:
  ``7/30/2024 12:20:23''
\item
  A special code or phrase that is likely to be unique to you but
  doesn't contain identifiable information.
\end{enumerate}

\end{tcolorbox}

\paragraph*{Wednesday, August 28}\label{wednesday-august-28}
\addcontentsline{toc}{paragraph}{Wednesday, August 28}

\begin{itemize}
\tightlist
\item
  Wrap up on
  \href{notes/wk01-2024-08-26-intro.qmd\#trust-in-science}{``Why trust
  science''}
\end{itemize}

\emph{Don't Fool Yourself}

\begin{itemize}
\tightlist
\item
  Read

  \begin{itemize}
  \tightlist
  \item
    (Feynman 1974).
  \item
    (Recommended) (Sagan 1996), Chapter 12, The Fine Art of Baloney
    Detection.
    \href{https://psu.instructure.com/courses/2350148/files/folder/readings?preview=165170710}{PDF
    on Canvas}
  \end{itemize}
\item
  \href{notes/wk01-2024-08-28-no-foolin.qmd}{Class notes}
\end{itemize}

\paragraph*{Friday, August 30}\label{friday-august-30}
\addcontentsline{toc}{paragraph}{Friday, August 30}

\emph{Work Session: How to read a scientific paper}

\begin{itemize}
\tightlist
\item
  Read

  \begin{itemize}
  \tightlist
  \item
    (Carey, Steiner, and Petri 2020)
  \item
    (Ruben 2016) (for fun)
  \end{itemize}
\item
  {Assignment}

  \begin{itemize}
  \tightlist
  \item
    \href{exercises/ex01-read-a-scientific-paper.qmd}{Exercise 01:
    Reading a scientific paper}
  \end{itemize}
\item
  \href{notes/wk01-2024-08-30-how-to-read.qmd}{Class notes}
\end{itemize}

\subsubsection*{September 2-6}\label{week-02}
\addcontentsline{toc}{subsubsection}{September 2-6}

\paragraph*{Monday, September 02}\label{monday-september-02}
\addcontentsline{toc}{paragraph}{Monday, September 02}

\textbf{NO CLASS, LABOR DAY}

\paragraph*{Wednesday, September 04}\label{wednesday-september-04}
\addcontentsline{toc}{paragraph}{Wednesday, September 04}

\emph{How science works (or should)}

\begin{itemize}
\tightlist
\item
  Read

  \begin{itemize}
  \tightlist
  \item
    (Ritchie 2020), Chapter 1.
    \href{https://psu.instructure.com/courses/2350148/files/folder/readings?preview=165170709}{Alternate
    link to PDF on Canvas}.
  \item
    (Brian A. Nosek and Bar-Anan 2012).
    \href{https://psu.instructure.com/courses/2350148/files/folder/readings?preview=165170711}{Alternate
    link to PDF on Canvas}
  \end{itemize}
\item
  {Assignment}

  \begin{itemize}
  \tightlist
  \item
    Complete (anonymous, extra credit)
    \href{https://forms.gle/zGm5vZsPrTxwRDVR7}{survey} on scientific
    norms and counter-norms. \textbf{No write-up}.
  \end{itemize}
\item
  \href{notes/wk02-2024-09-04-how-science-works.qmd}{Class notes}
\end{itemize}

\paragraph*{Friday, September 06}\label{friday-september-06}
\addcontentsline{toc}{paragraph}{Friday, September 06}

\textbf{Work Session: Reading a paper; Evaluating its claims}

\begin{itemize}
\tightlist
\item
  {Due}

  \begin{itemize}
  \tightlist
  \item
    \href{exercises/ex01-read-a-scientific-paper.qmd}{Exercise 01:
    Reading a scientific paper}
  \end{itemize}
\item
  {Assignment}

  \begin{itemize}
  \tightlist
  \item
    \href{exercises/ex02-textbook-findings.qmd}{Exercise 02: Textbook
    Findings}, {due Friday, September 20}
  \end{itemize}
\item
  \href{notes/wk02-2024-09-06-claims.qmd}{Class notes}, due
\end{itemize}

\subsubsection*{September 9-13}\label{week-03}
\addcontentsline{toc}{subsubsection}{September 9-13}

\paragraph{Monday, September 09}\label{monday-september-09}

\emph{Scientific norms and counter-norms}

\begin{itemize}
\tightlist
\item
  Read

  \begin{itemize}
  \tightlist
  \item
    (Merton 1973).
    \href{https://psu.instructure.com/courses/2350148/files/folder/readings?preview=165170713}{PDF
    on Canvas}.
  \item
    (Mitroff 1974).
    \href{https://psu.instructure.com/courses/2350148/files/folder/readings?preview=165170712}{PDF
    on Canvas}.
  \end{itemize}
\item
  {Assignment}

  \begin{itemize}
  \tightlist
  \item
    Complete (anonymous)
    \href{https://forms.gle/reRw9sYUzsYHUsqz9}{survey} on scientific
    norms and counter-norms. \textbf{No write-up}.
  \end{itemize}
\item
  \href{notes/wk03-2024-09-09-norms-counternorms.qmd}{Class notes}
\end{itemize}

\paragraph*{Wednesday, September 11}\label{wednesday-september-11}
\addcontentsline{toc}{paragraph}{Wednesday, September 11}

\emph{Adherence to norms and counter-norms}

\begin{itemize}
\tightlist
\item
  Read

  \begin{itemize}
  \tightlist
  \item
    \href{https://doi.org/10.1007/s11251-011-9195-0}{Kardash and Edwards
    (2012)}.
  \item
    \href{https://doi.org/10.1007/s10805-008-9055-y}{Macfarlane and
    Cheng (2008)}.
  \end{itemize}
\item
  \href{notes/wk03-2024-09-11-adherence-to-norms.qmd}{Class notes}
\end{itemize}

\paragraph*{Friday, September 13}\label{friday-september-13}
\addcontentsline{toc}{paragraph}{Friday, September 13}

\emph{Work session: Norms and counter-norms}

\begin{itemize}
\tightlist
\item
  Survey 02: \href{surveys/survey-02.qmd}{From raw data to results}
\item
  {Assignment}

  \begin{itemize}
  \tightlist
  \item
    \href{exercises/ex03-norms-counternorms.qmd}{Exercise 03: Norms and
    counter-norms}, {due Friday, September 27}
  \end{itemize}
\item
  \href{notes/wk03-2024-09-13-work-session-norms.qmd}{Class notes}
\end{itemize}

\subsubsection*{September 16-20}\label{week-04}
\addcontentsline{toc}{subsubsection}{September 16-20}

\paragraph*{Monday, September 16}\label{monday-september-16}
\addcontentsline{toc}{paragraph}{Monday, September 16}

\emph{A replication crisis (or not)}

\begin{itemize}
\tightlist
\item
  Read

  \begin{itemize}
  \tightlist
  \item
    (Ritchie 2020), Chapter 2.
    \href{https://psu.instructure.com/courses/2350148/files/folder/readings?preview=165170706}{PDF
    on Canvas}.
  \item
    (Begley and Ellis 2012).
  \end{itemize}
\item
  \href{notes/wk04-2024-09-16-replication-crisis.qmd}{Class notes}
\end{itemize}

\paragraph*{Wednesday, September 18}\label{wednesday-september-18}
\addcontentsline{toc}{paragraph}{Wednesday, September 18}

\emph{Replication attempt: The ``Lady Macbeth Effect''}

\begin{itemize}
\tightlist
\item
  Read

  \begin{itemize}
  \tightlist
  \item
    (Zhong and Liljenquist 2006).
  \item
    (Earp et al. 2014).
  \end{itemize}
\item
  \href{notes/wk04-2024-09-18-replication-lady-macbeth.qmd}{Class notes}
\end{itemize}

\paragraph*{Friday, September 20}\label{friday-september-20}
\addcontentsline{toc}{paragraph}{Friday, September 20}

\st{\emph{Work session: Replication with R, R Markdown, \& Quarto}}

\emph{Wrap-up on `Macbeth effect' replication}

\begin{itemize}
\tightlist
\item
  \st{Review}

  \begin{itemize}
  \tightlist
  \item
    \st{\href{surveys/survey-01.qmd}{Survey 01 report}}
  \item
    \st{\href{surveys/survey-02.qmd}{Survey 02 report}}
  \end{itemize}
\item
  {Due}

  \begin{itemize}
  \tightlist
  \item
    \href{exercises/ex02-textbook-findings.qmd}{Exercise 02: Textbook
    Findings}
  \end{itemize}
\item
  \href{notes/wk04-2024-09-20-work-session-R.qmd}{Class notes}
\end{itemize}

\subsubsection*{September 23-27}\label{week-05}
\addcontentsline{toc}{subsubsection}{September 23-27}

\paragraph*{Monday, September 23}\label{monday-september-23}
\addcontentsline{toc}{paragraph}{Monday, September 23}

\emph{Priming effect: Original study}

\begin{itemize}
\tightlist
\item
  Read

  \begin{itemize}
  \tightlist
  \item
    (Bargh, Chen, and Burrows 1996);
    \href{https://psu.instructure.com/courses/2350148/files/folder/readings?preview=165170715}{PDF
    on Canvas}
  \end{itemize}
\item
  \href{notes/wk05-2024-09-23-priming.qmd}{Class notes}
\end{itemize}

\paragraph*{Wednesday, September 25}\label{wednesday-september-25}
\addcontentsline{toc}{paragraph}{Wednesday, September 25}

\emph{Priming effect: Replication study}

\begin{itemize}
\tightlist
\item
  Read

  \begin{itemize}
  \tightlist
  \item
    (Doyen et al. 2012)
  \item
    review (Bargh, Chen, and Burrows 1996);
    \href{https://psu.instructure.com/courses/2350148/files/folder/readings?preview=165170715}{PDF
    on Canvas}
  \end{itemize}
\item
  \href{notes/wk05-2024-09-25-priming-replication.qmd}{Class notes}
\end{itemize}

\paragraph*{Friday, September 27}\label{friday-september-27}
\addcontentsline{toc}{paragraph}{Friday, September 27}

\emph{Work session: Scientific integrity \& Final Project Proposals}

\begin{itemize}
\tightlist
\item
  {Assignment}

  \begin{itemize}
  \tightlist
  \item
    \href{exercises/ex04-scientific-integrity.qmd}{Exercise 04:
    Violations of scientific integrity}, {due Friday, October 4}
  \item
    \href{exercises/final-project.qmd}{Final project proposals}, {due
    Friday, October 18}.
  \end{itemize}
\item
  {Due}

  \begin{itemize}
  \tightlist
  \item
    \href{exercises/ex03-norms-counternorms.qmd}{Exercise 03: Norms and
    counter-norms write-up}
  \end{itemize}
\item
  \href{notes/wk05-2024-09-27-work-integrity.qmd}{Class notes}
\end{itemize}

\subsubsection*{September 30 - October 4}\label{week-06}
\addcontentsline{toc}{subsubsection}{September 30 - October 4}

\paragraph{Monday, September 30}\label{monday-september-30}

\emph{Students' Choice}

\paragraph*{Wednesday, October 02}\label{wednesday-october-02}
\addcontentsline{toc}{paragraph}{Wednesday, October 02}

\emph{Fraud \& misconduct}

\begin{itemize}
\tightlist
\item
  Read

  \begin{itemize}
  \tightlist
  \item
    (Ritchie 2020), Chapter 3
  \item
    (Bhattacharjee 2013),
    \href{https://psu.instructure.com/courses/2350148/files/folder/readings?preview=165170718}{PDF
    on Canvas}
  \item
    (Skim) (Levelt, Drenth, and Noort 2012)
  \item
    (Skim) (Carpenter 2012)
  \end{itemize}
\end{itemize}

\paragraph*{Friday, October 04}\label{friday-october-04}
\addcontentsline{toc}{paragraph}{Friday, October 04}

\emph{Work session: P-hacking \& Final Project Proposals}

\emph{P-hacking}

\begin{itemize}
\tightlist
\item
  {Due}

  \begin{itemize}
  \tightlist
  \item
    \href{exercises/ex04-scientific-integrity.qmd}{Exercise 04:
    Violations of scientific integrity}
  \end{itemize}
\item
  {Assignment}

  \begin{itemize}
  \tightlist
  \item
    \href{exercises/ex05-p-hacking.qmd}{Exercise 05: P-hack your way to
    scientific glory} write-up.
  \end{itemize}
\item
  Work session

  \begin{itemize}
  \tightlist
  \item
    \href{exercises/final-project.qmd}{Final project proposals}, {due
    Friday, October 13}.
  \end{itemize}
\end{itemize}

\subsubsection*{October 7-11}\label{week-07}
\addcontentsline{toc}{subsubsection}{October 7-11}

\paragraph*{Monday, October 07}\label{monday-october-07}
\addcontentsline{toc}{paragraph}{Monday, October 07}

\emph{Retraction and scientific integrity}

\emph{On Zoom}: \url{https://psu.zoom.us/my/rogilmore}. Check-in for
attendance. Join from anywhere convenient to you.

\begin{itemize}
\tightlist
\item
  Read

  \begin{itemize}
  \tightlist
  \item
    (Brainerd and You 2018)
  \end{itemize}
\end{itemize}

\paragraph*{Wednesday, October 09}\label{wednesday-october-09}
\addcontentsline{toc}{paragraph}{Wednesday, October 09}

\textbf{NO CLASS}

\paragraph*{Friday, October 11}\label{friday-october-11}
\addcontentsline{toc}{paragraph}{Friday, October 11}

\emph{Questionable research practices}

\begin{itemize}
\tightlist
\item
  Read

  \begin{itemize}
  \tightlist
  \item
    (Simmons, Nelson, and Simonsohn 2011)
  \end{itemize}
\item
  Watch

  \begin{itemize}
  \tightlist
  \item
    (Ngiam 2020)
  \end{itemize}
\item
  Bring

  \begin{itemize}
  \tightlist
  \item
    Draft \href{exercises/ex05-p-hacking.qmd}{Exercise 05: P-hack your
    way to scientific glory} for discussion.
  \end{itemize}
\end{itemize}

\subsubsection*{October 14-18}\label{week-08}
\addcontentsline{toc}{subsubsection}{October 14-18}

\paragraph*{Monday, October 14}\label{monday-october-14}
\addcontentsline{toc}{paragraph}{Monday, October 14}

\emph{Work Session: P-hacking and Final Project Proposals}

\begin{itemize}
\tightlist
\item
  {Due}

  \begin{itemize}
  \tightlist
  \item
    \href{exercises/ex05-p-hacking.qmd}{Exercise 05: P-hack your way to
    scientific glory} write-up.
  \end{itemize}
\item
  Work session

  \begin{itemize}
  \tightlist
  \item
    \href{exercises/final-project.qmd}{Final project proposals}, {due
    Friday, October 18}.
  \end{itemize}
\end{itemize}

\paragraph*{Wednesday, October 16}\label{wednesday-october-16}
\addcontentsline{toc}{paragraph}{Wednesday, October 16}

\emph{Prevalence of QRPs}

\begin{itemize}
\tightlist
\item
  Read

  \begin{itemize}
  \tightlist
  \item
    (John, Loewenstein, and Prelec 2012)
  \end{itemize}
\item
  (Optional) Enter your \href{exercises/ex05-p-hacking.qmd}{Exercise 05}
  data into
  \href{https://docs.google.com/spreadsheets/d/1JI_Qih4wCzUrYTQYE3dpvVx2C7GdzfUZq0a3QhqQyeE/edit?usp=sharing}{this}
  anonymous spreadsheet.
\item
  Discuss \href{exercises/ex05-p-hacking.qmd}{Exercise 05: P-hack your
  way to scientific glory}.
\end{itemize}

\paragraph*{Friday, October 18}\label{friday-october-18}
\addcontentsline{toc}{paragraph}{Friday, October 18}

\emph{File drawer effect \& Work Session: Alpha, Power, Effect Sizes, \&
Sample Size}

\begin{itemize}
\tightlist
\item
  Read

  \begin{itemize}
  \tightlist
  \item
    (Skim) (Rosenthal 1979).
    \href{https://psu.instructure.com/courses/2350148/files/folder/readings?preview=165170719}{PDF
    on Canvas}
  \item
    (Optional) (Franco, Malhotra, and Simonovits 2014)
  \end{itemize}
\item
  {Assignment}

  \begin{itemize}
  \tightlist
  \item
    \href{exercises/ex06-apes.qmd}{Exercise 06: Alpha, Power, Effect
    Sizes, \& Sample Size}, {final version due Wednesday, October 30}
  \end{itemize}
\item
  {Due}

  \begin{itemize}
  \tightlist
  \item
    \href{exercises/final-project.qmd}{Final project} proposal
  \end{itemize}
\end{itemize}

\subsubsection*{October 21-25}\label{week-09}
\addcontentsline{toc}{subsubsection}{October 21-25}

\paragraph*{Monday, October 21}\label{monday-october-21}
\addcontentsline{toc}{paragraph}{Monday, October 21}

\emph{Negligence}

\begin{itemize}
\tightlist
\item
  Read

  \begin{itemize}
  \tightlist
  \item
    (Ritchie 2020), Chapter 5
  \item
    (Nuijten et al. 2015)
  \item
    (Szucs and Ioannidis 2017)
  \end{itemize}
\end{itemize}

\paragraph*{Wednesday, October 23}\label{wednesday-october-23}
\addcontentsline{toc}{paragraph}{Wednesday, October 23}

\emph{Hype}

\begin{itemize}
\tightlist
\item
  Read

  \begin{itemize}
  \tightlist
  \item
    (Ritchie 2020), Chapter 6
  \item
    (Carney, Cuddy, and Yap 2010),
    \href{https://psu.instructure.com/courses/2350148/files/folder/readings?preview=165170721}{file
    on Canvas}
  \item
    (Optional) (Ranehill et al. 2015),
    \href{https://psu.instructure.com/courses/2350148/files/folder/readings?preview=165170720}{file
    on Canvas}
  \end{itemize}
\end{itemize}

\paragraph*{Wednesday, October 23}\label{wednesday-october-23-1}
\addcontentsline{toc}{paragraph}{Wednesday, October 23}

\emph{Hype, continued}

\begin{itemize}
\tightlist
\item
  Watch \& discuss

  \begin{itemize}
  \tightlist
  \item
    \href{https://www.ted.com/talks/amy_cuddy_your_body_language_may_shape_who_you_are}{Cuddy
    (2012)}
  \end{itemize}
\end{itemize}

\paragraph*{Friday, October 25}\label{friday-october-25}
\addcontentsline{toc}{paragraph}{Friday, October 25}

\emph{Work session: Alpha, Power, Effect Sizes, \& Sample Size \&
Replication}

\begin{itemize}
\tightlist
\item
  {Discuss draft}

  \begin{itemize}
  \tightlist
  \item
    \href{exercises/ex06-apes.qmd}{Exercise 06: Alpha, Power, Effect
    Sizes, \& Sample Size} write-up
  \end{itemize}
\item
  {Assignment}

  \begin{itemize}
  \tightlist
  \item
    \href{exercises/ex07-replication.qmd}{Exercise 07: Replication},
    {due Friday, November 8}
  \end{itemize}
\item
  (Extra credit) Complete
  \href{https://forms.gle/BTqayMKnP7Xnyhst9}{Survey 03}
\end{itemize}

\subsubsection*{October 28 - November 1}\label{week-10}
\addcontentsline{toc}{subsubsection}{October 28 - November 1}

\paragraph*{Monday, October 28}\label{monday-october-28}
\addcontentsline{toc}{paragraph}{Monday, October 28}

\emph{Solutions}

\begin{itemize}
\tightlist
\item
  Read

  \begin{itemize}
  \tightlist
  \item
    (Munafò et al. 2017)
  \item
    (Begley 2013)
  \end{itemize}
\end{itemize}

\paragraph*{Wednesday, October 30}\label{wednesday-october-30}
\addcontentsline{toc}{paragraph}{Wednesday, October 30}

\emph{Changing journal policies}

\begin{itemize}
\tightlist
\item
  Read

  \begin{itemize}
  \tightlist
  \item
    (B. A. Nosek et al. 2015)
  \item
    (Gilmore et al. 2020)
  \item
    (SRCD 2019)
  \end{itemize}
\item
  {Final version due}

  \begin{itemize}
  \tightlist
  \item
    \href{exercises/ex06-apes.qmd}{Exercise 06: Alpha, Power, Effect
    Sizes, \& Sample Size} write-up
  \end{itemize}
\end{itemize}

\paragraph*{Friday, November 01}\label{friday-november-01}
\addcontentsline{toc}{paragraph}{Friday, November 01}

\emph{Catch-up day}

\subsubsection*{November 4-8}\label{week-11}
\addcontentsline{toc}{subsubsection}{November 4-8}

\paragraph*{Monday, November 04}\label{monday-november-04}
\addcontentsline{toc}{paragraph}{Monday, November 04}

\emph{Large-scale replication studies}

\begin{itemize}
\tightlist
\item
  Read

  \begin{itemize}
  \tightlist
  \item
    (Collaboration 2015)
  \end{itemize}
\end{itemize}

\paragraph*{Wednesday, November 06}\label{wednesday-november-06}
\addcontentsline{toc}{paragraph}{Wednesday, November 06}

\emph{Meta-analysis \& many analysts}

\begin{itemize}
\tightlist
\item
  Read

  \begin{itemize}
  \tightlist
  \item
    (Wilson 2014)
  \item
    (Silberzahn et al. 2018)
  \end{itemize}
\end{itemize}

\paragraph*{Friday, November 08}\label{friday-november-08}
\addcontentsline{toc}{paragraph}{Friday, November 08}

\emph{Work Session: Final Projects}

\begin{itemize}
\tightlist
\item
  {Due}

  \begin{itemize}
  \tightlist
  \item
    \href{exercises/ex07-replication.qmd}{Exercise 07: Replication}
  \end{itemize}
\end{itemize}

\subsubsection*{November 11-15}\label{week-12}
\addcontentsline{toc}{subsubsection}{November 11-15}

\paragraph*{Monday, November 11}\label{monday-november-11}
\addcontentsline{toc}{paragraph}{Monday, November 11}

\emph{Preregistration}

\begin{itemize}
\tightlist
\item
  Read

  \begin{itemize}
  \tightlist
  \item
    (Brian A. Nosek et al. 2018)
  \item
    (Ledgerwood 2018) or (Goldin-Meadow 2016)
  \item
    (Optional) (Claesen et al. 2021)
  \end{itemize}
\item
  Explore

  \begin{itemize}
  \tightlist
  \item
    \href{https://clinicaltrials.gov/}{clinicaltrials.gov}
  \end{itemize}
\end{itemize}

\paragraph*{Wednesday, November 13}\label{wednesday-november-13}
\addcontentsline{toc}{paragraph}{Wednesday, November 13}

\emph{Data sharing}

\begin{itemize}
\tightlist
\item
  Read

  \begin{itemize}
  \tightlist
  \item
    (Houtkoop et al. 2018)
  \item
    (Tenopir et al. 2020)
  \item
    (Optional) (Gilmore and Adolph 2017)
  \item
    (Optional) (Meyer 2018)
  \item
    (Optional)
    \href{https://grants.nih.gov/grants/guide/notice-files/NOT-OD-21-013.html}{National
    Institutes of Health (n.d.)}
  \end{itemize}
\end{itemize}

\paragraph*{Friday, November 15}\label{friday-november-15}
\addcontentsline{toc}{paragraph}{Friday, November 15}

\emph{Work Session: Final Projects}

\begin{itemize}
\tightlist
\item
  {Final project survey}

  \begin{itemize}
  \tightlist
  \item
    Please complete the \href{https://forms.gle/vtdJ82Y7AjgU15VA9}{final
    project survey} by {Friday, November 15}
  \end{itemize}
\end{itemize}

\subsubsection*{November 18-22}\label{week-13}
\addcontentsline{toc}{subsubsection}{November 18-22}

\paragraph*{Monday, November 18}\label{monday-november-18}
\addcontentsline{toc}{paragraph}{Monday, November 18}

\emph{Materials, code, \& protocol sharing}

\begin{itemize}
\tightlist
\item
  Read

  \begin{itemize}
  \tightlist
  \item
    (Soska et al. 2021)
  \item
    (Gilroy and Kaplan 2019)
  \end{itemize}
\item
  Explore

  \begin{itemize}
  \tightlist
  \item
    \href{https://www.protocols.io/}{protocols.io}
  \item
    \href{https://www.jove.com/}{Journal of Visualized Experiments
    (JOVE)}
  \end{itemize}
\item
  \href{}{Class notes}
\end{itemize}

\paragraph*{Wednesday, November 20}\label{wednesday-november-20}
\addcontentsline{toc}{paragraph}{Wednesday, November 20}

\emph{Open science tools}

\begin{itemize}
\tightlist
\item
  Read

  \begin{itemize}
  \tightlist
  \item
    (Kathawalla, Silverstein, and Syed 2021)
  \item
    {[}Chopik2018-wx{]}
  \item
    (Optional) (Crüwell et al. 2019)
  \end{itemize}
\item
  Explore

  \begin{itemize}
  \tightlist
  \item
    ({``{FORRT} - Framework for Open and Reproducible Research
    Training,''} n.d.)
  \end{itemize}
\end{itemize}

\paragraph*{Friday, November 22}\label{friday-november-22}
\addcontentsline{toc}{paragraph}{Friday, November 22}

\emph{Work session: Data sharing}

\begin{itemize}
\tightlist
\item
  {Assignment distributed}

  \begin{itemize}
  \tightlist
  \item
    \href{exercises/ex08-sharing.qmd}{Exercise 08: Data and materials
    sharing}
  \end{itemize}
\item
  {\href{https://forms.gle/vtdJ82Y7AjgU15VA9}{Final project survey}}
\end{itemize}

\subsubsection{\texorpdfstring{November 25-29 \emph{Thanksgiving
Break}}{November 25-29 Thanksgiving Break}}\label{november-25-29-thanksgiving-break}

\subsubsection*{December 2-6}\label{week-14}
\addcontentsline{toc}{subsubsection}{December 2-6}

\paragraph*{Monday, December 2}\label{monday-december-2}
\addcontentsline{toc}{paragraph}{Monday, December 2}

\emph{In-class final project work day}

\paragraph*{Wednesday, December 4}\label{wednesday-december-4}
\addcontentsline{toc}{paragraph}{Wednesday, December 4}

\emph{In-class final project work day}

\paragraph*{Friday, December 6}\label{friday-december-6}
\addcontentsline{toc}{paragraph}{Friday, December 6}

\emph{Project presentations}

\begin{itemize}
\tightlist
\item
  \href{}{Schedule}
\item
  {Assignment Due}

  \begin{itemize}
  \tightlist
  \item
    \href{exercises/ex08-sharing.qmd}{Exercise 08: Data and materials
    sharing}
  \end{itemize}
\end{itemize}

\subsubsection*{December 9-13}\label{week-15}
\addcontentsline{toc}{subsubsection}{December 9-13}

\paragraph*{Monday, December 9}\label{monday-december-9}
\addcontentsline{toc}{paragraph}{Monday, December 9}

\emph{Project presentations}

\begin{itemize}
\tightlist
\item
  \href{}{Schedule}
\end{itemize}

\paragraph*{Wednesday, December 11}\label{wednesday-december-11}
\addcontentsline{toc}{paragraph}{Wednesday, December 11}

\emph{Project presentations}

\begin{itemize}
\tightlist
\item
  \href{}{Schedule}
\end{itemize}

\paragraph*{Friday, December 13}\label{friday-december-13}
\addcontentsline{toc}{paragraph}{Friday, December 13}

\emph{The future of open, transparent, and reproducible scholarship}

\subsubsection*{December 16-20}\label{finals-week}
\addcontentsline{toc}{subsubsection}{December 16-20}

\paragraph*{Wednesday, December 18}\label{wednesday-december-18}
\addcontentsline{toc}{paragraph}{Wednesday, December 18}

\begin{itemize}
\tightlist
\item
  {Due}

  \begin{itemize}
  \tightlist
  \item
    \href{exercises/final-project.qmd}{Final project} write-ups due 5:00
    PM.
  \end{itemize}
\end{itemize}

\subsection{Student Evaluation}\label{student-evaluation}

\subsubsection*{Elements}\label{eval-components}
\addcontentsline{toc}{subsubsection}{Elements}

\begin{longtable}[]{@{}
  >{\raggedright\arraybackslash}p{(\columnwidth - 4\tabcolsep) * \real{0.1667}}
  >{\raggedright\arraybackslash}p{(\columnwidth - 4\tabcolsep) * \real{0.6970}}
  >{\raggedleft\arraybackslash}p{(\columnwidth - 4\tabcolsep) * \real{0.1364}}@{}}
\toprule\noalign{}
\begin{minipage}[b]{\linewidth}\raggedright
Component
\end{minipage} & \begin{minipage}[b]{\linewidth}\raggedright
Description
\end{minipage} & \begin{minipage}[b]{\linewidth}\raggedleft
Points
\end{minipage} \\
\midrule\noalign{}
\endhead
\bottomrule\noalign{}
\endlastfoot
Attendance & You will receive 1 point for each class you attend up to a
maximum of 40. & 40 \\
Exercises & There will be eight (8) exercises that you must work on.
Each exercise is worth 10 points. The top four (4) count toward your
final grade. & 40 \\
Final project & You will complete a
\href{exercises/final-project.qmd}{final project}, either on your own,
or with a small group of 3 or less. Your final project is worth 40
points. & 40 \\
& \textbf{TOTAL POINTS POSSIBLE} & 120 \\
Extra Credit & If you submit more than four exercise write-ups, you may
earn up to 10 extra credit points. & \\
\end{longtable}

\subsubsection*{Grading Scheme}\label{grading-scheme}
\addcontentsline{toc}{subsubsection}{Grading Scheme}

\begin{longtable}[]{@{}lcc@{}}
\toprule\noalign{}
Percent & Points & Grade \\
\midrule\noalign{}
\endhead
\bottomrule\noalign{}
\endlastfoot
94+ & 113 & A \\
90-93 & 108-112 & A- \\
87-89 & 104-107 & B+ \\
84-86 & 100-103 & B \\
80-83 & 96-99 & B- \\
77-79 & 92-95 & C+ \\
70-76 & 84-91 & C \\
60-69 & 72-93 & D \\
\textless59 & \textless=71 & F \\
\end{longtable}

\subsection{Important deadlines}\label{important-deadlines}

This page summarizes some of the key deadlines in the course.

\begin{longtable}[]{@{}
  >{\raggedright\arraybackslash}p{(\columnwidth - 2\tabcolsep) * \real{0.3333}}
  >{\raggedright\arraybackslash}p{(\columnwidth - 2\tabcolsep) * \real{0.6667}}@{}}
\toprule\noalign{}
\begin{minipage}[b]{\linewidth}\raggedright
Date
\end{minipage} & \begin{minipage}[b]{\linewidth}\raggedright
What's {due}/happening
\end{minipage} \\
\midrule\noalign{}
\endhead
\bottomrule\noalign{}
\endlastfoot
\href{schedule.qmd\#friday-september-06}{2024-09-06} &
\href{exercises/ex01-read-a-scientific-paper.qmd}{Exercise 01} \\
\href{schedule.qmd\#friday-september-20}{2024-09-20} &
\href{exercises/ex02-textbook-findings.qmd}{Exercise 02} \\
\href{schedule.qmd\#friday-september-27}{2024-09-20} &
\href{exercises/ex03-norms-counternorms.qmd}{Exercise 03} \\
\href{schedule.qmd\#friday-october-04}{2024-10-04} &
\href{exercises/ex04-scientific-integrity.qmd}{Exercise 04} \\
\href{schedule.qmd\#monday-october-14}{2024-10-14} &
\href{exercises/ex05-p-hacking.qmd\%22}{Exercise 05} \\
\href{schedule.qmd\#friday-october-18}{2024-10-18} &
\href{exercises/final-project.qmd}{Final project proposal} due \\
\href{schedule.qmd\#friday-november-01}{2024-11-01} &
\href{exercises/ex06-apes.qmd}{Exercise 06} \\
\href{schedule.qmd\#friday-november-08}{2024-11-08} &
\href{exercises/ex07-replication.qmd}{Exercise 07} \\
\href{schedule.qmd\#friday-november-22}{2024-11-22} &
\href{https://forms.gle/ajLqSsNfLiH2rJxi9}{Final Project Presentation
Schedule Survey} \\
\href{schedule.qmd\#friday-december-06}{2024-12-06} &
\href{exercises/ex08-sharing.qmd}{Exercise 08} \\
\href{schedule.qmd\#wednesday-december-18}{2024-12-18} &
\href{exercises/final-project.qmd}{Final project} writeup due \\
\end{longtable}

\subsection{Course policies}\label{course-policies}

\subsubsection*{Academic Integrity}\label{academic-integrity}
\addcontentsline{toc}{subsubsection}{Academic Integrity}

Students with questions about academic integrity should visit
\url{http://www.la.psu.edu/current-students/undergraduate-students/education/academic-integrity}.

Penn State defines academic integrity as the pursuit of scholarly
activity in an open, honest and responsible manner. All students should
act with personal integrity, respect others dignity, rights and
property, and help create and maintain an environment in which all can
succeed through the fruits of their efforts (Faculty Senate Policy
49-20). Sanctions for academic misconduct can include a grade of F for
the course as well as other penalties.

Unless you are told otherwise, you must complete all course work
entirely on your own, using only sources that have been permitted by
your instructor, and you may not assist other students with papers,
quizzes, exams, or other assessments. If I allow you to use ideas,
images, or word phrases created by another person (e.g., from Course
Hero or Chegg) or by generative technology, such as ChatGPT, you must
identify their source.

When you complete assignments, remember the \textbf{ABC}s to avoid
plagiarism: \textbf{A}lways place copied information within quotation
marks, include information about the quoted or paraphrased source in a
\textbf{B}ibliography, and \textbf{C}ite the source in the body (in the
text) of your paper immediately after the quoted or paraphrased
information. When in doubt, cite in the text and include the source in a
bibliography.

\textbf{Students with questions about academic integrity should ask me
or the TA before submitting work.}

Students facing allegations of academic misconduct may not drop/withdraw
from the affected course unless they are cleared of wrongdoing (see G-9:
Academic Integrity). Attempted drops will be prevented or reversed, and
students will be expected to complete course work and meet course
deadlines. Students who are found responsible for academic integrity
violations face academic outcomes, which can be severe, and put
themselves at jeopardy for other outcomes which may include
ineligibility for Dean's List, pass/fail elections, and grade
forgiveness. Students may also face consequences from their home/major
program and/or The Schreyer Honors College.

\subsubsection*{Absences or late
assignments}\label{late-missed-assignments}
\addcontentsline{toc}{subsubsection}{Absences or late assignments}

\paragraph{Absence from class}\label{absence-from-class}

Your absence from class may be excused under unusual circumstances such
as (a) an interview for graduate school or a job, (b) illness, (c)
religious observance, (d) the death of a family member, or (e) any other
event recognized by the university as a valid excuse for absence from
class.

If you must miss class, you must contact the instructor and the TA
\textbf{in advance}.

Up to three (3) excused absences will be permitted.

\paragraph*{Late exercises}\label{late-exercises}
\addcontentsline{toc}{paragraph}{Late exercises}

Exercises submitted after the published deadlines will not be eligible
for full credit unless the instructor has given specific permission.

\paragraph*{Late final projects}\label{late-final-projects}
\addcontentsline{toc}{paragraph}{Late final projects}

Final projects submitted after the published deadline will not be
eligible for full credit unless the instructor has given specific
permission.

\paragraph*{Accommodation for persons with
disabilities}\label{disability-accommodations}
\addcontentsline{toc}{paragraph}{Accommodation for persons with
disabilities}

Penn State welcomes students with disabilities into the University's
educational programs. Please refer to the information provided by
Student Disability Resources (SDR) at
\url{http://equity.psu.edu/student-disability-resources/} for
information about the procedures required to obtain reasonable
accommodations in this course. Students should discussSDR-approved
accommodations with their instructor as early in the semester as
possible, even if they have taken another course with the instructor.
Please note: students are not required to provide their instructor with
information about the nature of their condition.

Penn State students are also welcome to contact other units for
assistance with personal concerns that interfere with academic progress,
including: Counseling and Psychological Services (CAPS;
\url{http://studentaffairs.psu.edu/counseling/}), the Office of Student
Affairs (\url{http://studentaffairs.psu.edu/}), Career Services
(\url{http://studentaffairs.psu.edu/career/}), the Center for Women
Students (\url{http://studentaffairs.psu.edu/womenscenter/}), the LGBTQA
Student Resource Center (\url{http://studentaffairs.psu.edu/lgbtqa/}),
the Office of Sexual Misconduct Prevention and Response
(\url{http://titleix.psu.edu/}), Penn State Educational Equity
(\url{http://equity.psu.edu/}), the Multicultural Resource Center
(\url{http://equity.psu.edu/mrc}), and University Health Services
(\url{http://studentaffairs.psu.edu/health/}).

\subsubsection*{Nondiscrimination
Statement}\label{non-discrimination-statement}
\addcontentsline{toc}{subsubsection}{Nondiscrimination Statement}

The Pennsylvania State University is committed to equal access to
programs, facilities, admission and employment for all persons. It is
the policy of the University to maintain an environment free of
harassment and free of discrimination against any person because of age,
race, color, ancestry, national origin, religion, creed, service in the
uniformed services (as defined in state and federal law), veteran
status, sex, sexual orientation, marital or family status, pregnancy,
pregnancy-related conditions, physical or mental disability, gender,
perceived gender, gender identity, genetic information or political
ideas.\\
Discriminatory conduct and harassment, as well as sexual misconduct and
relationship violence, violates the dignity of individuals, impedes the
realization of the University's educational mission, and will not be
tolerated.

Direct all inquiries regarding the nondiscrimination policy to:

Dr.~Kenneth Lehrman III Vice Provost for Affirmative Action Affirmative
Action Office The Pennsylvania State University 328 Boucke Building
University Park, PA 16802-5901 Email: kfl2@psu.edu Tel (814) 863-0471

\subsubsection*{Diversity Statement}\label{diversity-statement}
\addcontentsline{toc}{subsubsection}{Diversity Statement}

This classroom is a place where you will be treated with respect. All
members of this class are expected to contribute to a respectful,
welcoming and inclusive environment for every other member of the
class.\\
Penn State is committed to creating an educational environment which is
free from intolerance directed toward individuals or groups and strives
to create and maintain an environment that fosters respect for others as
stated in Policy AD29 Statement on Intolerance.

\subsubsection*{Mandated Reporting Statement}\label{mandated}
\addcontentsline{toc}{subsubsection}{Mandated Reporting Statement}

Penn State's policies require me, as a faculty member, to share
information about incidents of sex-based discrimination and harassment
(discrimination, harassment, sexual harassment, sexual misconduct,
dating violence, domestic violence, stalking, and retaliation) with Penn
State's Title IX coordinator or deputy coordinators, regardless of
whether the incidents are stated to me in person or shared by students
as part of their coursework. For more information regarding the
University's policies and procedures for responding to reports of sexual
or gender-based harassment or misconduct, please visit
\url{http://titleix.psu.edu}.

Additionally, I am required to make a report on any reasonable suspicion
of child abuse in accordance with the Pennsylvania Child Protective
Services Law.

\subsubsection*{Zoom}\label{zoom}
\addcontentsline{toc}{subsubsection}{Zoom}

At some point in the semester, I may decide to use Zoom to allow
students who are unable to attend class in person to participate.

While you are on Zoom, keep in mind that this is a classroom environment
and others should be treated with respect. Please keep your microphone
muted unless you want to ask a question or interact with someone. If
your microphone is not muted, the entire class will be able to hear what
is going on in your environment. As an instructor, I personally like to
see people's faces. As a participant, I am more involved when I have my
camera on. I realize, however, that there are many reasons why you might
not want to turn on your camera such as poor internet connection,
joining via phone, or other privacy concerns. It is your choice as to
whether you would like to have the camera on or not.

\subsection{Values}\label{values}

\subsubsection*{Penn State Principles}\label{psu-principles}
\addcontentsline{toc}{subsubsection}{Penn State Principles}

The Pennsylvania State University is a community dedicated to personal
and academic excellence. The Penn State Principles were developed to
embody the values that we hope our students, faculty, staff,
administration, and alumni possess. At the same time, the University is
strongly committed to freedom of expression. Consequently, these
Principles do not constitute University policy and are not intended to
interfere in any way with an individual's academic or personal freedoms.
We hope, however, that individuals will voluntarily endorse these common
principles, thereby contributing to the traditions and scholarly
heritage left by those who preceded them, and will thus leave Penn State
a better place for those who follow.

\textbf{I will respect the dignity of all individuals within the Penn
State community}. The University is committed to creating and
maintaining an educational environment that respects the right of all
individuals to participate fully in the community. Actions motivated by
hate, prejudice, or intolerance violate this principle. I will not
engage in any behaviors that compromise or demean the dignity of
individuals or groups, including intimidation, stalking, harassment,
discrimination, taunting, ridiculing, insulting, or acts of violence. I
will demonstrate respect for others by striving to learn from
differences between people, ideas, and opinions and by avoiding
behaviors that inhibit the ability of other community members to feel
safe or welcome as they pursue their academic goals.

\textbf{I will practice academic integrity}. Academic integrity is a
basic guiding principle for all academic activity at Penn State
University, allowing the pursuit of scholarly activity in an open,
honest, and responsible manner. In accordance with the University Code
of Conduct, I will practice integrity in regard to all academic
assignments. I will not engage in or tolerate acts of falsification,
misrepresentation or deception because such acts of dishonesty violate
the fundamental ethical principles of the University community and
compromise the worth of work completed by others.

\textbf{I will demonstrate social and personal responsibility}. The
University is a community that promotes learning; any behaviors that are
inconsistent with that goal are unacceptable. Irresponsible behaviors,
including alcohol or drug abuse and the use of violence against people
or property, undermine the educational climate by threatening the
physical and mental health of members of the community. I will exercise
personal responsibility for my actions and I will make sure that my
actions do not interfere with the academic and social environment of the
University. I will maintain a high standard of behavior by adhering to
the Code of Conduct and respecting the rights of others.

\textbf{I will be responsible for my own academic progress and agree to
comply with all University policies}. The University allows students to
identify and achieve their academic goals by providing the information
needed to plan the chosen program of study and the necessary educational
opportunities, but students assume final responsibility for course
scheduling, program planning, and the successful completion of
graduation requirements. I will be responsible for seeking the academic
and career information needed to meet my educational goals by becoming
knowledgeable about the relevant policies, procedures, and rules of the
University and academic program, by consulting and meeting with my
adviser, and by successfully completing all of the requirements for
graduation.

\subsubsection*{Penn State Values}\label{psu-values}
\addcontentsline{toc}{subsubsection}{Penn State Values}

\href{http://universityethics.psu.edu/examples-action\#integrityexamples}{\textbf{Integrity}}:
We act with integrity and honesty in accordance with the highest
academic, professional, and ethical standards.

\href{http://universityethics.psu.edu/examples-action\#respectexamples}{\textbf{Respect}}:
We respect and honor the dignity of each person, embrace civil
discourse, and foster a diverse and inclusive community.

\href{http://universityethics.psu.edu/examples-action\#responsibilityexamples}{\textbf{Responsibility}}:
We act responsibly, and we are accountable for our decisions, actions,
and their consequences.

\href{http://universityethics.psu.edu/examples-action\#discoveryexamples}{\textbf{Discovery}}:
We seek and create new knowledge and understanding, and foster
creativity and innovation, for the benefit of our communities, society,
and the environment.

\href{http://universityethics.psu.edu/examples-action\#excellenceexamples}{\textbf{Excellence}}:
We strive for excellence in all our endeavors as individuals, an
institution, and a leader in higher education.

\href{http://universityethics.psu.edu/examples-action\#communityexamples}{\textbf{Community}}:
We work together for the betterment of our University, the communities
we serve, and the world.

\subsection*{References}\label{references}
\addcontentsline{toc}{subsection}{References}

\phantomsection\label{refs}
\begin{CSLReferences}{1}{0}
\bibitem[\citeproctext]{ref-Bargh1996-yv}
Bargh, J A, M Chen, and L Burrows. 1996. {``Automaticity of Social
Behavior: Direct Effects of Trait Construct and Stereotype-Activation on
Action.''} \emph{Journal of Personality and Social Psychology} 71 (2):
230--44. \url{https://doi.org/10.1037//0022-3514.71.2.230}.

\bibitem[\citeproctext]{ref-Begley2013-vm}
Begley, C Glenn. 2013. {``Six Red Flags for Suspect Work.''}
\emph{Nature} 497 (7450): 433--34.
\url{https://doi.org/10.1038/497433a}.

\bibitem[\citeproctext]{ref-Begley2012-cr}
Begley, C Glenn, and Lee M Ellis. 2012. {``Drug Development: Raise
Standards for Preclinical Cancer Research.''} \emph{Nature} 483 (7391):
531--33. \url{https://doi.org/10.1038/483531a}.

\bibitem[\citeproctext]{ref-Bhattacharjee2013-rw}
Bhattacharjee, Yudhijit. 2013. {``The Mind of a Con Man.''} \emph{The
New York Times}, April.
\url{https://www.nytimes.com/2013/04/28/magazine/diederik-stapels-audacious-academic-fraud.html}.

\bibitem[\citeproctext]{ref-Brainerd2018-iy}
Brainerd, Jeffrey, and Jia You. 2018. {``What a Massive Database of
Retracted Papers Reveals about Science Publishing's {`Death
Penalty'}.''} \emph{Science}, October.
\url{https://doi.org/10.1126/science.aav8384}.

\bibitem[\citeproctext]{ref-Carey2020-ut}
Carey, Maureen A, Kevin L Steiner, and William A Petri Jr. 2020. {``Ten
Simple Rules for Reading a Scientific Paper.''} \emph{PLoS
{C}omputational {B}iology} 16 (7): e1008032.
\url{https://doi.org/10.1371/journal.pcbi.1008032}.

\bibitem[\citeproctext]{ref-Carney2010-gq}
Carney, Dana R, Amy J C Cuddy, and Andy J Yap. 2010. {``Power Posing:
Brief Nonverbal Displays Affect Neuroendocrine Levels and Risk
Tolerance.''} \emph{Psychological Science} 21 (10): 1363--68.
\url{https://doi.org/10.1177/0956797610383437}.

\bibitem[\citeproctext]{ref-Carpenter2012-qy}
Carpenter, Siri. 2012. {``Harvard Psychology Researcher Committed Fraud,
{US} Investigation Concludes.''} \emph{Science} 6.
\url{https://www.science.org/content/article/harvard-psychology-researcher-committed-fraud-us-investigation-concludes}.

\bibitem[\citeproctext]{ref-Claesen2021-ae}
Claesen, Aline, Sara Gomes, Francis Tuerlinckx, and Wolf Vanpaemel.
2021. {``Comparing Dream to Reality: An Assessment of Adherence of the
First Generation of Preregistered Studies.''} \emph{Royal Society Open
Science} 8 (211037). \url{https://doi.org/10.1098/rsos.211037}.

\bibitem[\citeproctext]{ref-collaboration_estimating_2015}
Collaboration, Open Science. 2015. {``Estimating the Reproducibility of
Psychological Science.''} \emph{Science} 349 (6251): aac4716.
\url{https://doi.org/10.1126/science.aac4716}.

\bibitem[\citeproctext]{ref-Cruwell2019-nz}
Crüwell, Sophia, Johnny van Doorn, Alexander Etz, Matthew C Makel,
Hannah Moshontz, Jesse C Niebaum, Amy Orben, Sam Parsons, and Michael
Schulte-Mecklenbeck. 2019. {``Seven Easy Steps to Open Science.''}
\emph{Zeitschrift f{ü}r Psychologie} 227 (4): 237--48.
\url{https://doi.org/10.1027/2151-2604/a000387}.

\bibitem[\citeproctext]{ref-Cuddy2012-zx}
Cuddy, Amy. 2012. {``Your Body Language May Shape Who You Are.''}
\url{https://www.ted.com/talks/amy_cuddy_your_body_language_may_shape_who_you_are}.

\bibitem[\citeproctext]{ref-Doyen2012-ib}
Doyen, Stéphane, Olivier Klein, Cora-Lise Pichon, and Axel Cleeremans.
2012. {``Behavioral Priming: It's All in the Mind, but Whose Mind?''}
\emph{PloS One} 7 (1): e29081.
\url{https://doi.org/10.1371/journal.pone.0029081}.

\bibitem[\citeproctext]{ref-Earp2014-ek}
Earp, Brian D, Jim A C Everett, Elizabeth N Madva, and J Kiley Hamlin.
2014. {``Out, Damned Spot: Can the {`{M}acbeth Effect'} Be
Replicated?''} \emph{Basic and Applied Social Psychology} 36 (1):
91--98. \url{https://doi.org/10.1080/01973533.2013.856792}.

\bibitem[\citeproctext]{ref-Feynman1974-ld}
Feynman, R P. 1974. {``Cargo Cult Science.''}
\url{https://calteches.library.caltech.edu/51/2/CargoCult.htm}.
\url{https://calteches.library.caltech.edu/51/2/CargoCult.htm}.

\bibitem[\citeproctext]{ref-FORRT_undated-pl}
{``{FORRT} - Framework for Open and Reproducible Research Training.''}
n.d. \url{https://forrt.org/}. \url{https://forrt.org/}.

\bibitem[\citeproctext]{ref-Franco2014-yu}
Franco, Annie, Neil Malhotra, and Gabor Simonovits. 2014. {``Social
Science. Publication Bias in the Social Sciences: Unlocking the File
Drawer.''} \emph{Science} 345 (6203): 1502--5.
\url{https://doi.org/10.1126/science.1255484}.

\bibitem[\citeproctext]{ref-GilmoreAdolph2017}
Gilmore, Rick O, and Karen E Adolph. 2017. {``Video Can Make Behavioural
Research More Reproducible.''} \emph{Nature Human Behavior} 1.
\url{https://doi.org/10.1038/s41562-017-0128}.

\bibitem[\citeproctext]{ref-Gilmore2020-sl}
Gilmore, Rick O, Pamela M Cole, Suman Verma, Marcel A G Aken, and Carol
M Worthman. 2020. {``Advancing Scientific Integrity, Transparency, and
Openness in Child Development Research: Challenges and Possible
Solutions.''} \emph{Child Development Perspectives} 14 (1): 9--14.
\url{https://doi.org/10.1111/cdep.12360}.

\bibitem[\citeproctext]{ref-Gilroy2019-bf}
Gilroy, Shawn P, and Brent A Kaplan. 2019. {``Furthering Open Science in
Behavior Analysis: An Introduction and Tutorial for Using {GitHub} in
Research.''} \emph{Perspectives on Behavior Science} 42 (3): 565--81.
\url{https://doi.org/10.1007/s40614-019-00202-5}.

\bibitem[\citeproctext]{ref-Goldin-Meadow2016-mc}
Goldin-Meadow, S. 2016. {``Why Preregistration Makes Me Nervous.''}
\emph{APS Observer} 29 (7).
\url{https://www.psychologicalscience.org/observer/why-preregistration-makes-me-nervous}.

\bibitem[\citeproctext]{ref-Houtkoop2018-tl}
Houtkoop, Bobby Lee, Chris Chambers, Malcolm Macleod, Dorothy V M
Bishop, Thomas E Nichols, and Eric-Jan Wagenmakers. 2018. {``Data
Sharing in Psychology: A Survey on Barriers and Preconditions.''}
\emph{Advances in Methods and Practices in Psychological Science},
February, 2515245917751886.
\url{https://doi.org/10.1177/2515245917751886}.

\bibitem[\citeproctext]{ref-John2012-tk}
John, Leslie K, George Loewenstein, and Drazen Prelec. 2012.
{``Measuring the Prevalence of Questionable Research Practices with
Incentives for Truth Telling.''} \emph{Psychological Science} 23 (5):
524--32. \url{https://doi.org/10.1177/0956797611430953}.

\bibitem[\citeproctext]{ref-Kardash2012-kq}
Kardash, Carolanne M, and Ordene V Edwards. 2012. {``Thinking and
Behaving Like Scientists: Perceptions of Undergraduate Science Interns
and Their Faculty Mentors.''} \emph{Instructional Science} 40 (6):
875--99. \url{https://doi.org/10.1007/s11251-011-9195-0}.

\bibitem[\citeproctext]{ref-Kathawalla2021-tk}
Kathawalla, Ummul-Kiram, Priya Silverstein, and Moin Syed. 2021.
{``Easing into Open Science: A Guide for Graduate Students and Their
Advisors.''} \emph{Collabra. Psychology} 7 (1).
\url{https://doi.org/10.1525/collabra.18684}.

\bibitem[\citeproctext]{ref-Ledgerwood2018-sn}
Ledgerwood, Alison. 2018. {``The Preregistration Revolution Needs to
Distinguish Between Predictions and Analyses.''} \emph{Proceedings of
the National Academy of Sciences of the United States of America} 115
(45): E10516--17. \url{https://doi.org/10.1073/pnas.1812592115}.

\bibitem[\citeproctext]{ref-Levelt2012-ap}
Levelt, W J M, P J D Drenth, and E Noort. 2012. {``Flawed Science: The
Fraudulent Research Practices of Social Psychologist Diederik Stapel.''}
\url{https://pure.mpg.de/rest/items/item_1569964/component/file_1569966/content};
pure.mpg.de.
\url{https://pure.mpg.de/rest/items/item_1569964/component/file_1569966/content}.

\bibitem[\citeproctext]{ref-Macfarlane2008-tc}
Macfarlane, Bruce, and Ming Cheng. 2008. {``Communism, Universalism and
Disinterestedness: Re-Examining Contemporary Support Among Academics for
Merton's Scientific Norms.''} \emph{Journal of Academic Ethics} 6 (1):
67--78. \url{https://doi.org/10.1007/s10805-008-9055-y}.

\bibitem[\citeproctext]{ref-Merton1973-vf}
Merton, Robert W. 1973. {``The Normative Structure of Science.''} In
\emph{The {S}ociology of {S}cience: {T}heoretical and {E}mpirical
{I}nvestigations}, edited by Robert K Merton and Norman W Storer,
267--78. The University of Chicago Press.

\bibitem[\citeproctext]{ref-Meyer2018-vk}
Meyer, Michelle N. 2018. {``Practical Tips for Ethical Data Sharing.''}
\emph{Advances in Methods and Practices in Psychological Science},
February, 2515245917747656.
\url{https://doi.org/10.1177/2515245917747656}.

\bibitem[\citeproctext]{ref-Mitroff1974-pp}
Mitroff, Ian I. 1974. {``Norms and Counter-Norms in a Select Group of
the {A}pollo Moon Scientists: A Case Study of the Ambivalence of
Scientists.''} \emph{American Sociological Review} 39 (4): 579--95.
\url{https://doi.org/10.2307/2094423}.

\bibitem[\citeproctext]{ref-munafo_manifesto_2017}
Munafò, Marcus R., Brian A. Nosek, Dorothy V. M. Bishop, Katherine S.
Button, Christopher D. Chambers, Nathalie Percie du Sert, Uri Simonsohn,
Eric-Jan Wagenmakers, Jennifer J. Ware, and John P. A. Ioannidis. 2017.
{``A Manifesto for Reproducible Science.''} \emph{Nature Human
Behaviour} 1 (January): 0021.
\url{https://doi.org/10.1038/s41562-016-0021}.

\bibitem[\citeproctext]{ref-National_Institutes_of_Health_undated-it}
National Institutes of Health. n.d. {``{NOT-OD-21-013}: Final {NIH}
Policy for Data Management and Sharing.''}
\url{https://grants.nih.gov/grants/guide/notice-files/NOT-OD-21-013.html}.
\url{https://grants.nih.gov/grants/guide/notice-files/NOT-OD-21-013.html}.

\bibitem[\citeproctext]{ref-Ngiam2020-ln}
Ngiam, William. 2020. {``{ReproducibiliTea} \textbar{} Simmons, Nelson
and Simonsohn (2011). {False-Positive} Psychology.''} Youtube.
\url{https://www.youtube.com/watch?v=bf3GqyBRgzY}.

\bibitem[\citeproctext]{ref-nosek_promoting_2015}
Nosek, B. A., G. Alter, G. C. Banks, D. Borsboom, S. D. Bowman, S. J.
Breckler, S. Buck, et al. 2015. {``Promoting an Open Research
Culture.''} \emph{Science} 348 (6242): 1422--25.
\url{https://doi.org/10.1126/science.aab2374}.

\bibitem[\citeproctext]{ref-Nosek2012-al}
Nosek, Brian A, and Yoav Bar-Anan. 2012. {``Scientific Utopia i: Opening
Scientific Communication.''} \emph{Psychological {I}nquiry} 23 (3):
217--43. \url{https://doi.org/10.1080/1047840X.2012.692215}.

\bibitem[\citeproctext]{ref-Nosek2018-jk}
Nosek, Brian A, Charles R Ebersole, Alexander C DeHaven, and David T
Mellor. 2018. {``The Preregistration Revolution.''} \emph{Proceedings of
the National Academy of Sciences of the United States of America} 115
(11): 2600--2606. \url{https://doi.org/10.1073/pnas.1708274114}.

\bibitem[\citeproctext]{ref-Nuijten2015-ul}
Nuijten, Michéle B, Chris H J Hartgerink, Marcel A L M van Assen, Sacha
Epskamp, and Jelte M Wicherts. 2015. {``The Prevalence of Statistical
Reporting Errors in Psychology (1985--2013).''} \emph{Behavior Research
Methods}, October, 1--22.
\url{https://doi.org/10.3758/s13428-015-0664-2}.

\bibitem[\citeproctext]{ref-Oreskes2017}
Oreskes, Naomi. 2019. \emph{{W}hy {T}rust {S}cience}. Princeton
University Press.

\bibitem[\citeproctext]{ref-Ranehill2015-dj}
Ranehill, Eva, Anna Dreber, Magnus Johannesson, Susanne Leiberg, Sunhae
Sul, and Roberto A Weber. 2015. {``Assessing the Robustness of Power
Posing: No Effect on Hormones and Risk Tolerance in a Large Sample of
Men and Women.''} \emph{Psychological Science} 26 (5): 653--56.
\url{https://doi.org/10.1177/0956797614553946}.

\bibitem[\citeproctext]{ref-Ritchie2020-fm}
Ritchie, Stuart. 2020. \emph{Science Fictions: Exposing Fraud, Bias,
Negligence and Hype in Science}. 1st ed. Penguin Random House.
\url{https://www.amazon.com/Science-Fictions/dp/1847925669}.

\bibitem[\citeproctext]{ref-Rosenthal1979-zi}
Rosenthal, Robert. 1979. {``The File Drawer Problem and Tolerance for
Null Results.''} \emph{Psychological Bulletin} 86 (3): 638--41.
\url{https://doi.org/10.1037/0033-2909.86.3.638}.

\bibitem[\citeproctext]{ref-Ruben2016-om}
Ruben, Adam. 2016. {``How to Read a Scientific Paper.''}
\emph{Science\textbar{} AAAS {[}Internet{]}} 20.
\url{https://www.science.org/content/article/how-read-scientific-paper-rev2}.

\bibitem[\citeproctext]{ref-sagan-1996-baloney}
Sagan, Carl. 1996. \emph{The {D}emon-Haunted {W}orld: {S}cience as a
{C}andle in the {D}ark}. Ballantine Books.

\bibitem[\citeproctext]{ref-Silberzahn2018-st}
Silberzahn, R, E L Uhlmann, D P Martin, P Anselmi, F Aust, E Awtrey, Š
Bahník, et al. 2018. {``Many Analysts, One Data Set: Making Transparent
How Variations in Analytic Choices Affect Results.''} \emph{Advances in
Methods and Practices in Psychological Science} 1 (3): 337--56.
\url{https://doi.org/10.1177/2515245917747646}.

\bibitem[\citeproctext]{ref-simmons_false-positive_2011}
Simmons, Joseph P., Leif D. Nelson, and Uri Simonsohn. 2011.
{``False-Positive Psychology: {U}ndisclosed Flexibility in Data
Collection and Analysis Allows Presenting Anything as Significant.''}
\emph{Psychological Science} 22 (11): 1359--66.
\url{https://doi.org/10.1177/0956797611417632}.

\bibitem[\citeproctext]{ref-Soska2021-mh}
Soska, Kasey C, Melody Xu, Sandy L Gonzalez, Orit Herzberg, Catherine S
Tamis-LeMonda, Rick O Gilmore, and Karen E Adolph. 2021.
{``(Hyper)active Data Curation: A Video Case Study from Behavioral
Science.''} \emph{Journal of Escience Librarianship} 10 (3).
\url{https://doi.org/10.7191/jeslib.2021.1208}.

\bibitem[\citeproctext]{ref-Srcd2019-hg}
SRCD. 2019. {``Policy on Scientific Integrity, Transparency, and
Openness \textbar{} Society for Research in Child Development {SRCD}.''}
\url{https://www.srcd.org/policy-scientific-integrity-transparency-and-openness}.
\url{https://www.srcd.org/policy-scientific-integrity-transparency-and-openness}.

\bibitem[\citeproctext]{ref-Szucs2017-fc}
Szucs, Denes, and John P A Ioannidis. 2017. {``Empirical Assessment of
Published Effect Sizes and Power in the Recent Cognitive Neuroscience
and Psychology Literature.''} \emph{PLoS Biology} 15 (3): e2000797.
\url{https://doi.org/10.1371/journal.pbio.2000797}.

\bibitem[\citeproctext]{ref-Tenopir2020-mq}
Tenopir, Carol, Natalie M Rice, Suzie Allard, Lynn Baird, Josh Borycz,
Lisa Christian, Bruce Grant, Robert Olendorf, and Robert J Sandusky.
2020. {``Data Sharing, Management, Use, and Reuse: Practices and
Perceptions of Scientists Worldwide.''} \emph{PloS One} 15 (3):
e0229003. \url{https://doi.org/10.1371/journal.pone.0229003}.

\bibitem[\citeproctext]{ref-Wilson2014-ol}
Wilson, Laura C. 2014. {``Introduction to {Meta-Analysis}: A Guide for
the Novice.''}
\url{https://www.psychologicalscience.org/observer/introduction-to-meta-analysis-a-guide-for-the-novice}.
\url{https://www.psychologicalscience.org/observer/introduction-to-meta-analysis-a-guide-for-the-novice}.

\bibitem[\citeproctext]{ref-Zhong2006-nf}
Zhong, Chen-Bo, and Katie Liljenquist. 2006. {``Washing Away Your Sins:
{T}hreatened Morality and Physical Cleansing.''} \emph{Science} 313
(5792): 1451--52. \url{https://doi.org/10.1126/science.1130726}.

\end{CSLReferences}




\end{document}
